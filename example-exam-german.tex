%% example-exam-german.tex
%% Copyright 2021 Lukas Schmelzeisen
%
% This work may be distributed and/or modified under the
% conditions of the LaTeX Project Public License, either version 1.3
% of this license or (at your option) any later version.
% The latest version of this license is in
%   http://www.latex-project.org/lppl.txt
% and version 1.3 or later is part of all distributions of LaTeX
% version 2005/12/01 or later.
%
% This work has the LPPL maintenance status `maintained'.
% 
% The current maintainer of this work is Lukas Schmelzeisen.
%
% This work consists of the files acAssignment.cls and acAssignment.tex and
% bundled example files.


% Enable warnings about problematic code
\RequirePackage[l2tabu, orthodox]{nag}

\documentclass[exam, sectionseven, german]{acAssignment}

% Parse this file as UTF-8 (included in LaTeX by default since 2018 but included
% here for backwards-compatibility). If you use something else, change this.
\usepackage[utf8]{inputenc}

\acCourse[https://west.uni-koblenz.de/studying/ws1819/datenbanken]
    {Grundlagen der Datenbanken}
\acSemester{WS~18/19}
\acExamType{Hauptklausur} % Alternativ "Nachklausur" oder "Probeklausur"
\acExamDate{27. Februar 2019}

\acAuthor[https://west.uni-koblenz.de/de/about-us/team/claudia-schon]
    {Dr. Claudia Schon}{schon@uni-koblenz.de}
\acAuthor[https://west.uni-koblenz.de/de/about-us/team/daniel-janke]
    {Daniel Janke}{danijank@uni-koblenz.de}

\begin{document}
\maketitle

\acExamForm

\begin{itemize}[itemsep=0pt]
    \item Prüfen Sie, ob Ihr Klausurexemplar vollständig ist (\acNumPages~Seiten, \acNumTasks~Aufgaben).
    \item Verwenden Sie ein dokumentenechtes Schreibmedium, am besten Kugelschreiber.
        Schreiben Sie \emph{nicht} mit Bleistift.
    \item Die Nutzung von Hilfsmitteln jeglicher Art ist \emph{nicht gestattet}.
        Dies umfasst Taschenrechner, Mobiltelefone, Bücher, Notizen und vergleichbares.
    \item Sofern möglich, geben Sie Ihre Antworten bitte auf dem jeweiligen Aufgabenblatt an.
        Bei Bedarf können Sie weitere Blätter anfordern (3 sind bereits mit dieser Klausur am Ende gebündelt).
        Beschriften Sie diese mit Namen, Matrikel- und Aufgabennummer und referenzieren Sie diese vom jeweiligen Aufgabenblatt.
    \item Mit der Teilnahme an dieser Prüfung versichern Sie, dass Sie die Zulassungsvorraussetzungen für diese Prüfung erfüllen.
        Falls diese nicht erfüllt sind, wird die Prüfung als nicht abgelegt gewertet.
\end{itemize}

\vspace{0.1cm}
\textbf{Viel Erfolg!}

\vfill

\acNote:
Diese Klausur enthält keine Aufgaben, sondern wurde nur zum Testen erstellt.

% ==============================================================================
\part{Ohne Hilfsmittel}
Die folgenden Aufgaben müssen ohne Hilfsmittel bewältigt werden.

% ------------------------------------------------------------------------------
\section{Wissensfragen}
\acTask{1}

\subsection{Anfragesprachen}
\subsection{Indizes}
\subsection{Entwurfstheorie}
\subsection{Relationales Datenmodell}
\subsection{Modellierung}

% ------------------------------------------------------------------------------
\section{Modellierung}
\acTask{12}

% ------------------------------------------------------------------------------
\section{Relationale Entwurfstheorie}
\label{sec:rel-theorie}
Ein Verweis auf \cref{sec:rel-theorie}.

\subsection{Kandidatenschlüssel}
\label{subsec:candidatekey}
Ein Verweis auf \cref{subsec:candidatekey}.

\acTask{1}

\subsection{Normalformen}
\acTask{2.3}

\acTask{1.5}

\subsection{Synthesealgorithmus}
\acTask{12}

% ------------------------------------------------------------------------------
\section{SQL}

\subsection{Schuldirektion}
\acTask{3}

\subsection{Lehrervertretung}
\acTask{4}

\subsection{Bestenklasse}
\acTask{5}

% ------------------------------------------------------------------------------
\section{Logische Optimierung}
\acTask{10}

% ==============================================================================
\part{}
Auch für diesen Teil dürfen Sie keine Hilfsmittel verwenden.
Er dient nur als Beispiel für Klausuren mit mehreren Teilen.

% ------------------------------------------------------------------------------
\section{Kalküle \& Relationale Algebra}

\subsection{Muttersöhnchen}
\acTask{4}

\subsection{Backkönige}
\acTask{3}

\subsection{Buddies}
\acTask{5}

\subsection{Einseitige Ernährung}
\acTask{4}

% ------------------------------------------------------------------------------
\section{Erweiterbares Hashing}

\subsection{Hashing}
\acTask{4}

\subsection{Einfügen}
\acTask{8}

\acBlankPages{3}

\end{document}